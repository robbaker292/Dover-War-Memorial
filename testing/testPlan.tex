\documentclass[12pt]{article}
\usepackage[margin=1.5cm]{geometry}
\usepackage{wasysym} 
\usepackage{hyperref}

\renewcommand{\familydefault}{\sfdefault}

\newcounter{Test}
\newcommand{\test}[1]{%
\stepcounter{Test}%
\paragraph{\Circle\ Test \theTest} \textbf{#1} }

\title{\textbf{Dover War Memorial Project - Testing}}
\date{}
\author{}
%%%%% document start
\begin{document}

\maketitle

This document lists all the testing that must be taken place to ensure the site is bug free. I have done all these tests myself (and indeed found a number of bugs when doing so), but having been so closely involved in the design and construction of the site, someone else must also complete this. It will also allow you to spot any processes that you may find unusual, tricky, annoying or just incorrect. Any comments along the way on the actual processes, as well as the design and styling, are of great use. It is you, after all, who will be both using and maintaining the site. \smiley

\tableofcontents
\newpage

%\section{Introduction}

\section{Notes}
Buttons on screen are labelled in \textit{italics}, with text to be entered or displayed in \texttt{monospace}. Errors in links display a default 404 error page. This will be changed, but doing so would affect the Wedding website, so I haven't done that yet. The little circles before each test are for ticking to help see what is complete and what is not.

\section{Navigation}

These tests should ensure that the navigation is working as intended. If it is not, then the other tests might be a bit difficult(/annoying). These tests only need to be completed once, as the code is shared between all pages.

Navigate to \texttt{http://www.baker-wedding.org.uk/Dover-War-Memorial}. Note that the capitals are correct - this is a testing location, and on the real site, the address will be the same as before.
\test {Does the home page load correctly?}
Upon loading, the home page should be displayed. This will have the main title, followed by patrons, the top bar, the main blurb for the home page (contained within the pieces of paper effect), the bottom bar and copyright notice.

\test{Does the home page link work correctly? (Top)}
The left most button on the top bar should simply reload the page.

\test{Does the home page link work correctly? (Bottom)}
The left most button on the bottom bar should simply reload the page.

\test{Does the Latest News link work correctly? (Top)}
The second button on the top bar should load up the latest news for 2017. More on this in Section~\ref{sec:siteUpdate}.

\test{Does the Latest News link work correctly? (Bottom)}
The second button on the bottom bar should load up the latest news for 2017. More on this in Section~\ref{sec:siteUpdate}.

\test{Does the Casualty Index link work correctly? (Top)}
The third button on the top bar should load up the casualty index. More on this in Section~\ref{sec:memorial}.

\test{Does the Casualty Index link work correctly? (Bottom)}
The third button on the bottom bar should load up the casualty index. More on this in Section~\ref{sec:memorial}.

\test{Does the Articles link work correctly? (Top)}
The fourth button on the top bar should load up the information index (called Articles, as it's too long to fit in the box at the top). More on this in Section~\ref{sec:article}.

\test{Does the Articles link work correctly? (Bottom)}
The fourth button on the bottom bar should load up the information index. More on this in Section~\ref{sec:article}.

\test{Does the Search link work correctly? (Top)}
The fifth button on the top bar should load up the search page. More on this in Section~\ref{sec:search}.

\test{Does the Search link work correctly? (Bottom)}
The fifth button on the bottom bar should load up the search page. More on this in Section~\ref{sec:search}.

\test{Does the About \& Contact link work correctly? (Top)}
The sixth button on the top bar should load up the about page.

\test{Does the About \& Contact link work correctly? (Bottom)}
The sixth button on the bottom bar should load up the about page.

\test{Does the Login link work correctly? (Bottom)}
The last button on the bottom bar should load up the login form page. It is only displayed on the bottom page, as you are the only one who will need to login. More about this in Section~\ref{sec:user}.

\section{Login / Logout}\label{sec:user}
To edit various section of the site, you must be logged in. The username is \texttt{msk} and the password \texttt{hegerty}. The passwords are encrypted in the database, meaning they will be incredibly difficult to break. Tests in other sections will require some logging out, but I have written them to minimise the amount of logging out and back in again. To start, navigate to the login page using the \textit{Login} button in the bottom left.

\test{Does the login page forbid incorrect credentials?}
Enter an incorrect username and/or password and click \textit{Submit}. The page should reload with a message stating the username or password is incorrect. It will not inform the user which is incorrect - this is a standard security feature.

\test{Does the login page accept correct credentials?}
Enter the username and/or password and click \textit{Login}. The page should redirect to a page entitled \texttt{Meta Pages}. More on this in Section~\ref{sec:meta}.

\test{Have the admin controls appeared?}
Underneath the top menu bar another row should have appeared. It should indicate the currently logged in user, an admin button, and a button to logout (in red). The \textit{Login} button on the bottom bar should now be replaced by a \textit{Logout} button, also in red.

\test{Does the admin button work?}
Click the \textit{Admin Help} button in the middle left of the top bar. It should redirect to a page entitled \texttt{Admin Help}, which has very little content at present.

\test{Does the meta pages button work?}
Click the \textit{Meta Pages} button in the middle right of the top bar. It should redirect to a page entitled \texttt{Meta Pages}, which contains the .

\test{Does the Logout button work? (Top)}
Click the \textit{Logout} button on the right of the top bar. It should redirect back to the home page and the admin bar should now have disappeared.

\test{Does the Logout button work? (Bottom)}
Log back in again using the \textit{Login} button in the bottom bar. Once logged in, click the \textit{Logout} button on the right of the bottom bar. It should redirect back to the home page and the admin bar should now have disappeared. Stay logged in for now.

\section{Site Update}\label{sec:siteUpdate}
This section of the site contains two areas: Latest News, and a Change Log. Log in to the site if you have not already done so. Navigate to Site Update section, by clicking the \textit{Latest News} link in either menu bar.

\test{Have the latest news items been displayed?}
On the site update page, there should be a title, a link to view the change log and a list of latest news items. Each item should have a big title, some text and, in small font, the date it was posted. Next to each title, should be an \textit{Edit} button.

\test{Has the previous news summary been displayed?}
Further down the page, there should be a list of years alongside the corresponding number of updates for that year.

\test{Has the previous news summary been displayed correctly?}
Count the number of news items displayed on screen and ensure this matches the number shown for the current year.

\test{Are previous news items displayed correctly?}
Click on \textit{2016} to view the news from that year. There should be 2 items listed, the first being a long section of placeholder text entitled \texttt{Christmas}.

\test{Is an individual news item displayed correctly?}
Click on the \textit{Christmas} title. This item should now be displayed on it's own page.

\test{Is an invalid site update displayed gracefully?}
The URL should be similar to \texttt{/siteUpdate/view/11/Christmas}. Change the number after \texttt{view} to something very large, e.g. \texttt{5000}. It does not matter about the last section of the address, this can actually be anything. Instead of loading site number \texttt{5000} (as it does not exist), the home page should be displayed instead. This will occur in normal action if someone has a link to a deleted site update, or has incorrectly typed a link, or is trying to mess about.

\test{Is the return button correct?}
Ensure that the button at the bottom of the page reads \texttt{Return to all 2016 updates}. Click on this. The list of news items from 2016 should now be shown, as it was on a previous test.

\test{Is the list of news items displayed correctly when there are no updates?}
Navigate back to the 2017 list, either by selecting the date from the bottom, or by clicking the \textit{Latest News} button in either menu bar. Change the URL in the address bar, replacing \texttt{2017}, with some other date (for example, \texttt{2027}). The page should reload with a message indicating there are no updates for that year.

\test{Is there an Add button?}
Navigate back to the 2017 list, either by selecting the date from the bottom, or by clicking the \textit{Latest News} button in either menu bar. Next to the title should be displayed, in green, a button labelled \textit{New}.

\test{Is there an New page?}
Click this button. There should now be a page displayed entitled \texttt{New Update}. Enter a title, narrative and the date of the update (choose some time in the past in 2017). Do not worry about the narrative buttons at present, these will be tested in Section~\ref{sec:other}. Also enter some text in the \texttt{What's changed} box. Make a note of what you have entered here, we will use this in a later test.

\test{Has the site update been added?}
When complete, click the \textit{Save update} button. The page should now be redirected to the list of updates for 2017, and the newly added update should be displayed at the top.

\test{Does the New page stop invalid values? (1)}
Click the \textit{New} button again to be taken to the page entitled \texttt{New Update}. Without entering anything, click the \textit{Save update} button. The page should not reload and should instead highlight the title and narrative boxes, displaying \texttt{This field is required} below them.

\test{Does the New page stop invalid values? (2)}
Enter some text in the \texttt{Title} field, but leaving the \texttt{Narrative} field blank. Click the \textit{Save update} button. The page again should not reload but instead highlight only the narrative box.

\test{Does the New page stop invalid values? (3)}
Remove the text from the \texttt{Title} field and instead add text into the \texttt{Narrative} field. Click the \textit{Save update} button. The page again should not reload but instead highlight only the title box.

\test{Is there an Edit page?}
Return to the list of updates for 2017, by going Back in your browser, or clicking the \textit{Latest News} button in either menu bar. Displayed next to each news item should be a \textit{Edit}. Click on this button for the site update created earlier. A page should load that is similar to the New page seen earlier, but this time with some boxes complete.

\test{Does the View button work?}
Next to the title of this site update, there should be a button labelled \textit{View}. Click this and the site update in question should be loaded.

\test{Has the edit been saved?}
Return to the edit page, by clicking the \textit{Edit} button next to the title of this site update. Change some of the details and when complete, click \textit{Save update}. The list of updates for 2017 should now be displayed and the changes to the recently edited site update should be shown.

\test{Is an invalid site update edit page displayed gracefully?}
Return to the list of updates for 2017, by going Back in your browser, or clicking the \textit{Latest News} button in either menu bar. Displayed next to each news item should be a \textit{Edit}. Click on this button for the site update created earlier. The URL should be similar to \texttt{/siteUpdate/edit/11}. Change the number after \texttt{edit} to something very large, e.g. \texttt{5000}. Instead of loading site number \texttt{5000} (as it does not exist), the list of site updates for 2017 should be displayed instead. This will occur in normal action if someone has a link to a deleted site update, or has incorrectly typed a link, or is trying to mess about.

\test{Does a Site Update have a delete button?}
Create a new site update using the process described earlier. This update should now appear on the list of those in 2017. Click the \textit{Edit} button next to its title to be taken to the edit page. In the top right corner, a button in red should read \textit{Delete Site Update}.

\test{Can a deletion be stopped?}
Click this \textit{Delete Site Update} button. A pop-up warning should be displayed. Click \textit{Cancel}. The popup should be removed.

\test{Can a site update be deleted?}
Click the \textit{Delete Site Update} button. A pop-up warning should be displayed. This time, click \textit{Delete}. All deletions cannot be undone. The page should redirect to the list of updates for 2017, with the newly deleted update missing.

\test{Have the latest change log items been displayed?}
On the list of updates of a given year, there is a button to \textit{View change log}. Click this button. The page now loads a list of changes to the site in the current year. These are any edits or additions to the site, where the \texttt{What's changed} box has been completed. This allows users to view what is being changed without a news article being posted. Completing the \texttt{What's changed} on an edit is optional, and probably not worth it for a small typo.

\test{Has the previous change log items summary been displayed?}
Further down the page, there should be a list of years alongside the corresponding number of change log entries for that year.

\test{Has the previous change log items summary been displayed correctly?}
Count the number of news change log entries displayed on screen and ensure this matches the number shown for the current year.

\test{Is the change log displayed correctly when there are no updates?}
Change the URL in the address bar, replacing \texttt{2017}, with some other date (for example, \texttt{2027}). If no 2017 is shown, add \texttt{/2027} to the end of the URL. The page should reload with a message indicating there are no updates for that year.

\test{Is an individual news item displayed correctly?}
Navigate back to the change log for 2017, either by the Back button in your browser, or by clicking the \textit{Latest News} button in either menu bar and then the \textit{View change log} button. One of the most recent entries in the change log will be the creation of a news item described in an earlier test. The \texttt{Description} column should match the text that was entered then.

\test{Does the View button work correctly?}
Click the arrow button next to the relevant entry in the change log. The news item created earlier should now be displayed.

\test{Does a change log entry have a delete button?}
Return to the change log for 2017, either by the Back button in your browser, or by clicking the \textit{Latest News} button in either menu bar and then the \textit{View change log} button. For the change log entry created used earlier, it should have a red dustbin button under the \texttt{Delete} column.

\test{Can a deletion be stopped?}
Click this \textit{Delete} button. A pop-up warning should be displayed. Click \textit{Cancel}. The popup should be removed.

\test{Can a site update be deleted?}
Click the \textit{Delete} button. A pop-up warning should be displayed. This time, click \textit{Delete}. All deletions cannot be undone. Not that this only removes the change log entry, not the edit, nor the item in question. The page should reload with that change log item removed.

\test{Are site updates with posted dates in the future hidden in the main list?}
Using the method described earlier, create a new site update, ensuring that the posted date is some time in the future. Navigate back to the list of site updates. This newly created update should not be visible in the main list.

\test{Are site updates with posted dates in the future shown in the future list?}
Navigate to the main config page by using the \textit{Config \& Meta Pages} link in the top bar. There will be more tests on this page in Section~\ref{sec:meta}. Navigate down to the list entitled \texttt{Future Site Updates}. The newly created site update should be visible here.

\test{Are the Edit/New buttons removed when logged out?}
For the following tests, you will need to be logged out. Navigate back to the Latest News section. The \textit{New} button next to the title and \textit{Edit} buttons next to each title should no longer be displayed.

\test{Are the Delete buttons removed when logged out on the Change Log?}
For the following tests, you will need to be logged out. Navigate back to the Change Log. The dustbin buttons next to each entry should no longer be displayed.

\section{Memorial}\label{sec:memorial}
This section of the site contains the list of memorials and some data associated with them. It also contains some of the major new functionality - the memorial map. Log in to the site if you have not already done so.

\test{Has the memorial index been displayed?}
Click on the \textit{Casualty Index} button on the top or bottom menu bars. Displayed should be a page entitled \texttt{Casualty Index}. This will have a quote, followed by a button and two sections - one showing some memorials, the other collapsed.

\test{Has the main memorial list been displayed?}
After the main section of text, there should be a list entitled \texttt{Memorials}. This will list all the main memorials (all those on the original site outside the "More memorials" list), and the number of casualties associated with each one.

\test{Does the other memorial list been displayed?}
After the previous list, there should be a collapsible list showing the other memorials. Click the title of this should show and hide this list.

\test{Does the a memorial get displayed correctly?}
On the main list of memorials, click on \textit{Dover War Memorial}. This should load a page on that particular memorial, entitled \texttt{Dover War Memorial}. It should have a description of the memorial and a worded description of the location.

\test{Does the map get displayed correctly?}
On the Dover War Memorial page, there should be a map in the top right. A marker should be displayed in the correct location. Zooming in may be required to identify this more closely.

\test{Does the list of casualties get displayed?}
At the bottom of the page, there should be a series of buttons. The first row should read \texttt{World War One}, \texttt{World War Two}, and \texttt{All}. The second row should list all letters, plus an asterisk (indicating other characters). Displayed on screen should be all casualties whose surname begins with an A. There should be 3 casualties, all with improbable names.

\test{Does the list of casualties get updated?}
Clicking \textit{World War Two} should leave only one casualty displayed, while clicking \textit{World War One} should display two. Selecting \textit{World War One} and \textit{S} should result in a long list of casualties displayed.

\test{Does the list of casualties gracefully show if there are no casualties?}
Select \textit{World War One} and \textit{S} should result in a message stating that no results have been found.

\test{Is an invalid memorial displayed gracefully?}
The URL should be similar to \texttt{/memorial/view/1/Dover+War+Memorial}. Change the number after \texttt{view} to something very large, e.g. \texttt{5000}. It does not matter about the last section of the address, this can actually be anything. Instead of loading memorial number \texttt{5000} (as it does not exist), the home page should be displayed instead. This will occur in normal action if someone has a link to a deleted memorial, or has incorrectly typed a link, or is trying to mess about.

\test{Does the memorial map get displayed?}
Return to the casualty index, either by clicking back on your browser, or using the \textit{Casualty Index} button in either the top or bottom menu bars. From here, click the \textit{View Memorial Map} button. This should load a page entitled \texttt{Memorial Map}, containing a large map, centred on Dover. There should be two markers visible, one in the centre of Dover and another half way towards Folkestone.

\test{Does the information get displayed about a memorial?}
Click on the marker located over the centre of Dover. Below the map should be displayed the name of that memorial and the number of casualties commemorated there.

\test{Does the information link correctly?}
Click on the name of the memorial displayed (in this case, \texttt{Dover War Memorial}). This should now load the memorial page for the Dover War Memorial, as seen in an earlier test.

\test{Is there an Add button?}
Return to the casualty index, either by clicking back on your browser, or using the \textit{Casualty Index} button in either the top or bottom menu bars. Next to the title, there should be a \textit{New Memorial} button.

\test{Is there a New page?}
Click this button. There should now be a page displayed entitled \texttt{New Memorial}. Enter a title and narrative. Do not worry about the narrative buttons at present, these will be tested in Section~\ref{sec:other}. In the \texttt{latitude} box, enter \texttt{51.078324} and in the \texttt{longitude} box enter \texttt{1.187405}. This should place the memorial in Folkestone Harbour. Leave the \texttt{Order in main section} box blank. There is no need to complete the \texttt{What's changed} box.

\test{Has the memorial been added?}
When complete, click the \textit{Save memorial} button. The page should now be redirected to the list of memorials. The newly created memorial should be in the list under \textit{Other Memorials}. It should have zero casualties.

\test{Has the memorial's map data been added correctly? (1)}
Click on the name of the newly created memorial. The map in the top right should have a marker placed over Folkestone Harbour.

\test{Has the memorial's map data been added correctly? (2)}
Return to the casualty index, either by clicking back on your browser, or using the \textit{Casualty Index} button in either the top or bottom menu bars, then click on \textit{View Memorial Map}. The large map should load, with a new marker placed over Folkestone Harbour. 

\test{Does the New page stop invalid values? (1)}
Return to the casualty index, either by clicking back on your browser, or using the \textit{Casualty Index} button in either the top or bottom menu bars. Click the \textit{New Memorial} button again to be taken to the page entitled \texttt{New Memorial}. Without entering anything, click the \textit{Save memorial} button. The page should not reload and should instead highlight the name and content boxes, displaying \texttt{This field is required} below them.

\test{Does the New page stop invalid values? (2)}
Enter some text in the \texttt{Name} field, but leaving the \texttt{Content} field blank. Click the \textit{Save memorial} button. The page again should not reload but instead highlight only the narrative box.

\test{Does the New page stop invalid values? (3)}
Remove the text from the \texttt{Name} field and instead add text into the \texttt{Content} field. Click the \textit{Save memorial} button. The page again should not reload but instead highlight only the title box.

\test{Is there an edit page?}
Return to the casualty index, either by clicking back on your browser, or using the \textit{Casualty Index} button in either the top or bottom menu bars. The newly created memorial should be in the list under \textit{Other Memorials}. Click on the title of this memorial. The page for this memorial should load. Next to the title, is an \textit{Edit} button. Click this. A page should load that is similar to the new page seen earlier, except with some fields now complete.

\test{Does the View button work?}
Next to the title of this memorial, there should be a button labelled \textit{View}. Click this and the memorial in question should be loaded.

\test{Has the edit been saved?}
Return to the edit page, by clicking the \textit{Edit} button next to the title of this site update. Change some of the details and when complete, click \textit{Save memorial}. The list of memorials should now be displayed. Click \textit{Other memorials} to view the other memorials and select the newly edited memorial. The page for this should load and the changes can be seen.

\test{Does a Memorial have a delete button?}
Create a new memorial using the process described earlier. This memorial will now get added to one of the two lists. Click the memorial to view it, and then click the \textit{Edit} button next to its title to be taken to the edit page. In the top right corner, a button in red should read \textit{Delete Memorial}.

\test{Can a deletion be stopped?}
Click this \textit{Delete Memorial} button. A pop-up warning should be displayed. Click \textit{Cancel}. The popup should be removed.

\test{Can a memorial be deleted?}
Click the \textit{Delete Memorial} button. A pop-up warning should be displayed. This time, click \textit{Delete}. All deletions cannot be undone. The page should redirect to the list of memorials, with the newly deleted memorial missing. Any casualties who were commemorated here have had that piece of data removed.

\test{Is an invalid memorial page displayed gracefully?}
Navigate back to the list of memorials and click on any memorial. The URL should be similar to \texttt{memorial/view/1/Dover+War+Memorial}. Change the number after \texttt{view} to something very large, e.g. \texttt{5000}. It does not matter about the last section of the address, this can actually be anything. Instead of loading memorial number \texttt{5000} (as it does not exist), the home page should be displayed instead. This will occur in normal action if someone has a link to a deleted memorial, or has incorrectly typed a link, or is trying to mess about.

\test{Is an invalid edit page for a memorial displayed gracefully?}
Navigate back to the list of memorials and click on any memorial. Then, click the \textit{Edit} button on that memorial. The URL should be similar to \texttt{memorial/edit/1}. Change the number after \texttt{edit} to something very large, e.g. \texttt{5000}. Instead of loading the edit page for memorial number \texttt{5000} (as it does not exist), the list of memorials should be displayed instead.

\test{Are the Edit/New buttons removed when logged out?}
For the following tests, you will need to be logged out. Navigate back to the Casualty Index section. The \textit{New Memorial} button next to the title should no longer be displayed. Click on a memorial. There should be no \textit{Edit} button next to the title.

\section{Casualty}\label{sec:casualty}
The casualty section is heavily linked within the memorials. Casualties contain large amounts of information which is used to create the database and for the new search system. Log in to the site if you have not already done so.

\test{Is the casualty summary displayed correctly?}
Return to the casualty index by using the \textit{Casualty Index} button in either the top or bottom menu bars, then click on a memorial - \textit{Dover War Memorial} is a good one to choose. The memorial place should load. Displayed should be the three casualties with the surname A. Their names should be shown, along with their date of death.

\test{Is a casualty displayed correctly?}
Click on a casualty's name, for example \texttt{Frank Aaamith}. The page for this casualty should be loaded. The title should be the casualty's name, with the narrative displayed below.

\test{Are a casualty's details displayed correctly?}
There should also be a hidden table entitled \texttt{Casualty Details} at the bottom of the page. Clicking on this should load a table full of details. The casualty's given, middle and family names are shown, along with other data if available. Fields that have no data are not displayed. A number of these should be links, but they will be tested in Section~\ref{sec:search}.

\test{Is there an Add button?}
Return to the casualty index, either by clicking back on your browser, or using the \textit{Casualty Index} button in either the top or bottom menu bars. Next to the title, there should be a \textit{New Casualty} button.

\test{Is there a New page?}
Click this button. There should now be a page displayed entitled \texttt{New Casualty}. Enter details in the relevant boxes. Do not worry about the narrative buttons at present, these will be tested in Section~\ref{sec:other}. A number of boxes have drop down options. The contents of these options will be tested in Section~\ref{sec:other}.

\test{Do the other data options appear?}
When complete, click the \textit{Save and continue with editing} button. This will allow the addition of other data that relies on the casualty being saved and created first. The page should refresh with more fields available.

\test{Can a service number be added?}
In the \texttt{Service Number} section, enter a number (or text) into the \texttt{Add number} field and click \textit{Add number}. This should then appear in the drop down above. Click the dropdown and remove the empty entry.

\test{Can relation data be added?}
In the \texttt{Relation Date} section, click \textit{Add another relation}. This should load two drop down boxes. In the first, select the name of another casualty and in the second.

\test{Has the casualty been added?}
First, ensure that the Commemorations Locations is set to one or more locations. When editing is complete, click the \textit{Save all sections} button at the foot of the page. The page should display a green section indicating the Save was complete.

\test{Has the casualty been added correctly? (1)}
Click on the button labelled \textit{View} next to the casualty's name in the title. The casualty page will now load, display the casualty's name and narrative. Clicking the \textit{Casualty Details} hidden table will display all the data entered.

\test{Has the casualty been added correctly? (2)}
In the \textit{Casualty Details} table, there should be an entry in the \texttt{Relations} selection. Click the name of the casualty displayed. On that casualty's page, ensure that the first casualty is also displayed.

\test{Has the casualty been added correctly? (3)}
Return to the casualty index by using the \textit{Casualty Index} button in either the top or bottom menu bars, then click on the memorial that the newly created casualty is commemorated at. Using the buttons, select the relevant war and surname letter and the casualty, along with their date of death should be shown.

\test{Does the New page stop invalid values? (1)}
Return to the casualty index, either by clicking back on your browser, or using the \textit{Casualty Index} button in either the top or bottom menu bars. Click the \textit{New Casualty} button again to be taken to the page entitled \texttt{New Casualty}. Without entering anything, click the \textit{Save and continue with editing} button. The page should not reload and should instead highlight the family name and narrative boxes, displaying \texttt{This field is required} below them.

\test{Does the New page stop invalid values? (2)}
Enter some text in the \texttt{Family Name} field, but leaving the \texttt{Narrative} field blank. Click the \textit{Save and continue with editing} button. The page again should not reload but instead highlight only the narrative box.

\test{Does the New page stop invalid values? (3)}
Remove the text from the \texttt{Family Name} field and instead add text into the \texttt{Narrative} field. Click the \textit{Save and continue with editing} button. The page again should not reload but instead highlight only the title box.

\test{Is there an edit page?}
Next to the title of the casualty, is an \textit{Edit} button. Click this. A page should load that is similar to the new page seen earlier, except with some fields now complete.

\test{Can a service number be removed?}
There should be at lease one service number selected from a previous test. Deselect this one in the dropdown.

\test{Can relation data be removed?}
There should be at least one relation selected from a previous test. In both drop down boxes, deselect the chosen options.

\test{Has the edit been saved?}
Change some of the details and when complete, click \textit{Save all sections}. The page should display a green section indicating the Save was complete.

\test{Has the edit been saved correctly?}
Click on the button labelled \textit{View} next to the casualty's name in the title. The casualty page will now load, display the casualty's name and narrative. Clicking the \textit{Casualty Details} hidden table will display all the data entered and any changes can be seen.

\test{Does changing the \texttt{Recently Imported} field affect the display? (1)}
Return to the edit page of a given casualty. Ensure that the \texttt{Recently Imported} field is set to \texttt{Yes}. Click \textit{Save and return to casualty}. The casualty page will load. Clicking on the \textit{Casualty Details} hidden table will show the casualty's details. Displayed at the top of this table is a message informing the user that some details may not yet have been added.

\test{Does changing the \texttt{Recently Imported} field affect the display? (2)}
Using the \textit{Config \& Meta Pages} link on the top bar, find the \textit{View All} button further down the page in the section entitled \texttt{Recently Uploaded Casualties}. Click this button. There should be a list shown which contains the casualty that has just been modified.

\test{Does changing the \texttt{Recently Imported} field affect the display? (3)}
Return to the edit page of a given casualty. Ensure that the \texttt{Recently Imported} field is set to \texttt{No}. Click \textit{Save and return to casualty}. There should now be no message displayed at the top of the  \textit{Casualty Details} hidden table for this casualty.

\test{Does changing the \texttt{Recently Imported} field affect the display? (4)}
Return to the list of recently imported casualties, by using the \textit{Config \& Meta Pages} link on the top bar, find the \textit{View All} button further down the page in the section entitled \texttt{Recently Uploaded Casualties}. Click this button. The list should no longer contain the casualty that was just modified.

\test{Does changing the \texttt{Unsure Details} field affect the display? (1)}
Return to the edit page of a given casualty. Ensure that the \texttt{Unsure Details} field is set to \texttt{Yes}. Click \textit{Save and return to casualty}. The casualty page will load. Clicking on the \textit{Casualty Details} hidden table will show the casualty's details. Displayed at the top of this table is a message informing the user that some details may be unknown. This replacing the previous method of placing an asterisk next to a surname.

\test{Does changing the \texttt{Unsure Details} field affect the display? (2)}
Return to the edit page of a given casualty. Ensure that the \texttt{Unsure Details} field is set to \texttt{No}. Click \textit{Save and return to casualty}. There should now be no message displayed at the top of the \textit{Casualty Details} hidden table for this casualty.

\test{Does a casualty have a delete button?}
Create a new casualty using the process described earlier. This casualty will now get added to a memorial. Click the memorial and select the casualty from the filter buttons. When the casualty's page loads, click the \textit{Edit} button next to its title to be taken to the edit page. In the top right corner, a button in red should read \textit{Delete Casualty}.

\test{Can a deletion be stopped?}
Click this \textit{Delete Casualty} button. A pop-up warning should be displayed. Click \textit{Cancel}. The popup should be removed.

\test{Can a casualty be deleted?}
Click the \textit{Delete Casualty} button. A pop-up warning should be displayed. This time, click \textit{Delete}. All deletions cannot be undone. The page should redirect to the list of memorials.

\test{Is an invalid casualty page displayed gracefully?}
Navigate back to the list of casualties and click on any casualty. The URL should be similar to \texttt{casualty/view/1/Frank-Aaamith}. Change the number after \texttt{view} to something very large, e.g. \texttt{5000}. It does not matter about the last section of the address, this can actually be anything. Instead of loading casualty number \texttt{5000} (as it does not exist), the home page should be displayed instead. This will occur in normal action if someone has a link to a deleted casualty, or has incorrectly typed a link, or is trying to mess about.

\test{Is an invalid edit page for a casualty displayed gracefully?}
Navigate back to the list of casualties and click on any casualty. Then, click the \textit{Edit} button on that casualty. The URL should be similar to \texttt{casualty/edit/1}. Change the number after \texttt{edit} to something very large, e.g. \texttt{5000}. Instead of loading the edit page for casualty number \texttt{5000} (as it does not exist), the home page should be displayed instead.

\test{Is a casualty with no commemoration location displayed in the correct list?}
Create a new casualty using the method described earlier, ensuring that the casualty has no commemoration location set. Navigate to the config page by clicking the \textit{Config \& Meta pages} link in the top menu bar. From this, there should be a list entitled \texttt{Casualties With No Memorial}. In this list, the newly created casualty should be shown.

\test{Are the Edit/New buttons removed when logged out?}
For the following tests, you will need to be logged out. Navigate back to the Casualty Index section. The \textit{New Casualty} button next to the title should no longer be displayed. Click on a memorial and click on any casualty. There should be no \textit{Edit} button next to their name.

\test{Is the list of recently uploaded casualties visible when logged out?}
Ensure that you are logged out for this test. Navigate to \texttt{casualty/uploaded}. The page should not load and instead, the home page should be shown.

\section{Articles}\label{sec:article}
This section is the ``Information Index'' on the previous site, the shorter name fitting more nicely into the evenly spaced menu bar. Information that was on other areas of the site (e.g. the article about the Dover War Memorial) should be classed as an article, but won't belong in one of the groups that existed on the previous site. Log in to the site if you have not already done so.

\test{Is the article list displayed?}
Return to the article list by clicking on the \textit{Articles} button on either menu bar. There should be a title, a small piece of text, and several sections with lists of articles.

\test{Is the article list displayed correctly?}
Each section of the article list should have at least one entry. There should be no sections without entries, as these are not shown.

\test{Is an article displayed correctly?}
Click on an article's name, for example \texttt{Zeebrugge}. The page should load the article. The title will be at the top, with the main section of the article displayed below. The posted date should be in small text at the bottom.

\test{Is there an Add button?}
Return to the article list, either by clicking back on your browser, or using the \textit{Articles} button in either the top or bottom menu bars. Next to the title, there should be a \textit{New} button.

\test{Is there a New page?}
Click this button. There should now be a page displayed entitled \texttt{New Article}. Enter details in the relevant boxes. Do not worry about the narrative buttons at present, these will be tested in Section~\ref{sec:other}. Make sure that this article has a category.

\test{Has the article been added?}
When editing is complete, click the \textit{Save article} button at the foot of the page. The page should reload to the list of articles.

\test{Has the article been added correctly?}
Click on the name of the recently added article, which should be in the correct section. The article should load with the correct title, narrative and date entered.

\test{Does the New page stop invalid values? (1)}
Return to the article list, either by clicking back on your browser, or using the \textit{Article} button in either the top or bottom menu bars. Click the \textit{New} button again to be taken to the page entitled \texttt{New Article}. Without entering anything, click the \textit{Save article} button. The page should not reload and should instead highlight the family name and narrative boxes, displaying \texttt{This field is required} below them.

\test{Does the New page stop invalid values? (2)}
Enter some text in the \texttt{Title} field, but leaving the \texttt{Content} field blank. Click the \textit{Save article} button. The page again should not reload but instead highlight only the narrative box.

\test{Does the New page stop invalid values? (3)}
Remove the text from the \texttt{Title} field and instead add text into the \texttt{Content} field. Click the \textit{Save article} button. The page again should not reload but instead highlight only the title box.

\test{Is there an edit button?}
Return to the article list, either by clicking back on your browser, or using the \textit{Article} button in either the top or bottom menu bars, and click on the name of an article. Next to the title of the article, is an \textit{Edit} button.

\test{Is there an edit page?}
Click this \textit{Edit} button. A page should load that is similar to the new page seen earlier, except with some fields now complete.

\test{Has the edit been saved?}
Change some of the details and when complete, click \textit{Save article}. The page should return to the list of articles

\test{Has the edit been saved correctly?}
Click on the name of the article that has just been edited. The article will now load, displaying the article's title and content, with the changes shown.

\test{Does the View button work correctly?}
Click on the button labelled \textit{Edit} next to the article's title. The edit page should load. When it has, click the \textit{View} button. This should load the page for the article that was being edited.

\test{Is an invalid article page displayed gracefully?}
Navigate back to the list of articles and click on any article. The URL should be similar to \texttt{article/view/1/Zeebrugge}. Change the number after \texttt{view} to something very large, e.g. \texttt{5000}. It does not matter about the last section of the address, this can actually be anything. Instead of loading article number \texttt{5000} (as it does not exist), the home page should be displayed instead. This will occur in normal action if someone has a link to a deleted article, or has incorrectly typed a link, or is trying to mess about.

\test{Is an invalid edit page for an article displayed gracefully?}
Navigate back to the list of articles and click on any article. Then, click the \textit{Edit} button on that article. The URL should be similar to \texttt{article/edit/1}. Change the number after \texttt{edit} to something very large, e.g. \texttt{5000}. Instead of loading the edit page for article number \texttt{5000} (as it does not exist), the list of articles should be displayed instead.

\test{Does an article have a delete button?}
Create a new article using the process described earlier. Click the title of the newly created article. When the article's page loads, click the \textit{Edit} button next to its title to be taken to the edit page. In the top right corner, a button in red should read \textit{Delete Article}.

\test{Can a deletion be stopped?}
Click this \textit{Delete Article} button. A pop-up warning should be displayed. Click \textit{Cancel}. The popup should be removed.

\test{Can an article be deleted?}
Click the \textit{Delete Article} button. A pop-up warning should be displayed. This time, click \textit{Delete}. All deletions cannot be undone. The page should redirect to the list of articles.

\test{Is an article with no category displayed in the correct list?}
Create a new article, but assign it no category. Then navigate to the config page by clicking the \textit{Config \& Meta pages} link on the top menu bar. In the section entitled \texttt{Articles With No Category}, the newly created article should be shown.

\test{Are the Edit/New buttons removed when logged out?}
For the following tests, you will need to be logged out. Navigate back to the Articles section. The \textit{New} button next to the title should no longer be displayed. Click on an article. There should be no \textit{Edit} button next to the article title.

\section{Meta}\label{sec:meta}
Meta pages are the ``other'' pages of the site. These come in two varieties: 1) Standalone pages (e.g. Contact) or 2) Sections of Text (e.g. at the top of the Articles list or the Home page). The idea of these is that is allows you to create (or edit) pages that do not fit into any other category and link to these as needed. It also allows the editing of narrative sections on pages that would otherwise be bland lists. There are a few meta pages that are required - for example the top of the Home page, Contact Us, etc. These have all been added into the site already.

It is not possible to delete a meta page. There is a good reason for this. If a page was created that is no longer needed, simply remove any links to that page. If a page was accidentally deleted that was integral to the site (the Home page, for instance), that would be a disaster. I can't imagine that these pages will need to be created or edited much, if ever.

Tests in this section may require URLs to be entered. Log in to the site if you have not already done so.

\test{Is the list of Meta Pages displayed correctly?}
Navigate to the list of Meta Pages, by clicking on the \textit{Meta Pages} button on the top menu bar. The list of Meta pages should load. Their should be a small narrative section, followed by two lists. This page allows the editing of various information used in several places on the site (e.g. the list of Wars), as well as displaying the list of other meta pages. The top list will be tested later in Section~\ref{sec:other}.

\test{Is a meta page displayed correctly? (1)}
From the bottom list, click on \texttt{contactUs}. This should display the Contact Us page, with a small panel at the bottom indicating where this page is used.

\test{Is a meta page displayed correctly? (2)}
Navigate to the \textit{About \& Contact Us} page by using the buttons on either the top and bottom menu bars. The Contact Us page should load in a similar manner to the previous test, without the explanation panel.

\test{Is there an Add button?}
Return back to the list of Meta Pages, by clicking on the \textit{Meta Pages} button on the top menu bar. Next to the title (\texttt{Meta Pages}), there should be a button labelled \textit{New}.

\test{Is there a New page?}
Click this button. There should now be a page displayed entitled \texttt{New Meta Page}. Enter details in the relevant boxes. The \texttt{ID} field should have a value that does not match any of the other meta pages. Do not worry about the narrative buttons at present, these will be tested in Section~\ref{sec:other}. The \texttt{What's changed} section can be left empty - in normal operation any ``internal'' pages should never have this field changed, otherwise it will appear in the change log.

\test{Has the meta page been added?}
When editing is complete, click the \textit{Save meta page} button at the foot of the page. The page should reload to the list of meta pages.

\test{Has the meta page been added correctly?}
Click on the newly added meta page on the bottom list. The page should load with the created meta page, and the various details as entered.

\test{Does the New page stop invalid values? (1)}
Return back to the list of Meta Pages, by clicking on the \textit{Meta Pages} button on the top menu bar. Click the \textit{New} button again to be taken to the page entitled \texttt{New Meta Page}. Without entering anything, click the \textit{Save meta page} button. The page should not reload and should instead highlight the \texttt{ID}, \texttt{Title}, \texttt{Content} and \texttt{Where is this page used?} fields, displaying \texttt{This field is required} below each one. 

\test{Does the New page stop invalid values? (2)}
Enter some text in the \texttt{Content} box, leaving the other still empty. Click the \textit{Save meta page} button. The page should again not reload but still highlight the \texttt{ID}, \texttt{Title} and \texttt{Where is this page used?} fields.

\test{Is there an edit button? (1)}
Return back to the list of Meta Pages, by clicking on the \textit{Meta Pages} button on the top menu bar. The piece of text at the top of the \texttt{Meta Pages} link is itself a meta page. This page should have a button displayed below \textit{Edit This Text}.

\test{Is there an edit button? (2)}
Click one of the meta pages on the bottom list. This should display the meta page. On this page, next to the title should be a button entitled \textit{Edit}.

\test{Is there an edit page?}
Click this \textit{Edit} button. A page should load that is similar to the new page seen earlier, except with some fields now complete.

\test{Has the edit been saved?}
Change some of the details (content is a good idea) and when complete, click \textit{Save meta page}. The page should return to the list of meta pages.

\test{Has the edit been saved correctly? (1)}
The changed text should now be displayed at the top of this page.

\test{Has the edit been saved correctly? (2)}
Click on the name of the changed meta page on the list at the bottom. The meta page should load with the changed details.

\test{Is an invalid meta page displayed gracefully?}
Return back to the list of Meta Pages, by clicking on the \textit{Meta Pages} button on the top menu bar. Click on any meta page. The URL should be similar to \texttt{meta/viewFull/admin}. The last word will be the unique ID of the page. Change this to a value that has not already been used. Instead of loading this page that does not exist, the home page should be shown instead.

\test{Is an invalid edit page for a meta page displayed gracefully?}
Return back to the list of Meta Pages, by clicking on the \textit{Meta Pages} button on the top menu bar. Click on any meta page, followed by the \textit{Edit} button. The URL should be similar to \texttt{meta/edit/admin}. The last word will be the unique ID of the page. Change this to a value that has not already been used. Instead of loading the edit page for a meta page that does not exist, the home page should be shown instead.

\test{Is the most recent news item displayed on the home page?}
Create a site update news item using the method described above. Ensure that the date makes it the most recent news item (but not in the future). Return to the home page by using the \textit{Home} buttons on either menu bar. This newly created site update should appear in the area at the bottom left on the page.

\test{Are the deaths today displayed on the home page?}
Create a casualty using the method described above. Ensure the \texttt{Date of Death} is set to today's day and month, but in a previous year. Return to the home page by using the \textit{Home} buttons on either menu bar. The list of \texttt{Casualties commemorated today} section in the bottom right should now have this casualty listed along with their age. Clicking on the name of the casualty should load their casualty page.

\test{Are the Edit buttons removed when logged out?}
For the following tests, you will need to be logged out. Navigate to the home page, by using the \textit{Home} button on either menu bar. There should be no \textit{Edit} button displayed next to the title page.

\test{Is the meta page list visible when logged out?}
Navigate to \texttt{meta/listAll}. The page should not load as before and instead redirect to the home page.

\section{Search}\label{sec:search}

The searching capabilities of the site have dramatically improved. There are three main ways of searching: by text, by data and from a casualty. The first allows specific pieces of text to be found in a casualty's narrative, an article, site update or memorial. The second way allows for the searching of a specific piece of data, for example all those who were in the Army. The third way is from the data list on a casualty. Clicking on a link there will create a search based on that piece of data.

You do not need to be logged in to perform these tests. The first block of tests require that you are on the search page. Navigate here using the \textit{Search} link on either the top or bottom menu bar. 

\test{Does the search work for text and a category?}
In the first box, enter the search term \texttt{Zeebrugge} and in the second box, select \texttt{Casualty}. Then, click \textit{Search}. This should load a section further down the page entitled \textit{Search results}, with one result, a \texttt{F H Sole}.

\test{Are the search results displayed correctly?}
The \texttt{Content} column for the first result should show a section of the particular narrative with the search term highlighted. Clicking on the casualty's name should load up the casualty page for \texttt{F H Sole}.

\test{Does the search work for text and multiple categories?}
Return to the search page. Again in the first box, type in \texttt{Zeebrugge}. In the second, select \texttt{Casualty} and \texttt{Article} and click \textit{Search}. Two results should be displayed, the casualty previously mentioned and an article entitled \texttt{Zeebrugge}.

\test{Does the search page gracefully handle text but no category?}
Keep the same text in the first box, but deselect all options in the second and click \textit{Search}. The page should reload with all options empty.

\test{Does the search page gracefully handle a category but no text?}
This time, enter no text but select a category and click \textit{Search}. Again, the page should reload with all options empty.

\test{Does the pagination work correctly?}
Return to the search page if needed. In the first box, enter \texttt{Dover} and select \texttt{Casualty} in the second. Click \textit{Search}. Many results should be displayed towards the bottom of the screen, along with a list of page numbers. Click one of these. A different set of results should be shown.

\test{Does the search page gracefully handle having no results?}
In the first box, type a search term that will not be found. This could be a mash of letters, but leave the second as \texttt{Casualty}. Click \textit{Search}. Where the results have been displayed previously, instead should be a messaging stating that no results have been found.

\test{Does the data search work for a casualty's Given Name?}
Load a blank search page by clicking the \textit{Search} button on either the top of bottom menu bars. To load the data search, click the area entitled \textit{Search Casualty by data}. In the \texttt{Given Name} field, type \texttt{Albert} and click \textit{Search at the bottom of this section}. One result should be shown, a casualty named \texttt{Albert Henry Samways}.

\test{Does the data search work for a casualty's Middle Name?}
Load a blank search page by clicking the \textit{Search} button on either the top of bottom menu bars. In the data section, in the \texttt{Middle Names} field, type \texttt{Henry} and click \textit{Search} at the bottom of the page. Again, one result should be shown and again it should be \texttt{Albert Henry Samways}.

\test{Does the data search work for a casualty's Family Name?}
Load a blank search page by clicking the \textit{Search} button on either the top of bottom menu bars. In the data section, in the \texttt{Family Name} field, type \texttt{Thompson} and click \textit{Search} at the bottom of the page. Several results should be shown, all of whom should have the surname \texttt{Thompson}.

\test{Does the data search work for a casualty's Civilian status?}
Load a blank search page by clicking the \textit{Search} button on either the top of bottom menu bars. In the data section, select \texttt{No} in the \texttt{Civilian} box and click \textit{Search}. Many results will be returned.

\test{Does the data search work for a casualty's Gender?}
Load a blank search page by clicking the \textit{Search} button on either the top of bottom menu bars. In the data section, select \texttt{Male} in the \texttt{Gender} box and click \textit{Search}. Many results will be returned.

\test{Does the data search work for a casualty's Date of Birth}
Load a blank search page by clicking the \textit{Search} button on either the top of bottom menu bars. In the data section, in the \texttt{Date of Birth} field, enter \texttt{1898-10-17} box and click \textit{Search}. This will return at least one result, a casualty called \texttt{Frederick Thompson}.

\test{Does the data search work for a casualty's Date of Death}
Load a blank search page by clicking the \textit{Search} button on either the top of bottom menu bars. In the data section, in the \texttt{Date of Death} field, enter \texttt{1915-02-03} box and click \textit{Search}. This will return at least one result, a casualty called \texttt{Frederick Thompson}.

\test{Does the data search work for a casualty's War?}
Load a blank search page by clicking the \textit{Search} button on either the top of bottom menu bars. In the data section, select \texttt{World War I} in the \texttt{War} box and click \textit{Search}. Many casualties will be returned.

\test{Does the data search work for a casualty's Final Resting Place?}
Load a blank search page by clicking the \textit{Search} button on either the top of bottom menu bars. In the data section, select \texttt{Dover Cemetery} in the \texttt{Final Resting Place} box and click \textit{Search}. At least one casualty will be returned.

\test{Does the data search work for a casualty's Rank?}
Load a blank search page by clicking the \textit{Search} button on either the top of bottom menu bars. In the data section, select \texttt{Private} in the \texttt{Rank} box and click \textit{Search}. At least one casualty will be returned.

\test{Does the data search work for a casualty's Service Country?}
Load a blank search page by clicking the \textit{Search} button on either the top of bottom menu bars. In the data section, select \texttt{United Kingdom} in the \texttt{Service Country} box and click \textit{Search}. At least one casualty will be returned.

\test{Does the data search work for a casualty's Place of Birth}
Load a blank search page by clicking the \textit{Search} button on either the top of bottom menu bars. In the data section, select \texttt{Dover Hospital} in the \texttt{Place of Birth} box and click \textit{Search}. At least one casualty will be returned.

\test{Does the data search work for a casualty's Last Known Address?}
Load a blank search page by clicking the \textit{Search} button on either the top of bottom menu bars. In the data section, select \texttt{23 London Road} in the \texttt{Last Known Address} box and click \textit{Search}. At least one casualty will be returned.

\test{Does the data search work for a casualty's Last Known Address Year?}
Load a blank search page by clicking the \textit{Search} button on either the top of bottom menu bars. In the data section, type \texttt{1911} in the \texttt{Last Known Address Year} box and click \textit{Search}. At least one casualty will be returned.

\test{Does the data search work for a casualty's Commemoration Location?}
Load a blank search page by clicking the \textit{Search} button on either the top of bottom menu bars. In the data section, select \texttt{Dover War Memorial} in the \texttt{Commemoration Location} box and click \textit{Search}. Many casualties will be returned.

\test{Does the data search work for a casualty's Regiment / Service?}
Load a blank search page by clicking the \textit{Search} button on either the top of bottom menu bars. In the data section, select \texttt{Army} in the \texttt{Regiment/Service} box and click \textit{Search}. At least one casualty will be returned.

\test{Does the data search work for a casualty's Service Numbers?}
Load a blank search page by clicking the \textit{Search} button on either the top of bottom menu bars. In the data section, select any of the \texttt{Service Numbers} and click \textit{Search}. At least one casualty will be returned.

\test{Does the casualty data link to the correct search page? (Given Name)}
Data can also be search from a casualty. Navigate to the casualty called \texttt{Frank Aaamith}, as this testing entry has all the data fields supplied. Under the data section of this casualty, click on the given name. This should load the search page, showing other casualties with the same given name.

\test{Does the casualty data link to the correct search page? (Middle names)}
Navigate back to the casualty named \texttt{Frank Aaamith}. Under the data section of this casualty, click on the middle name. This should load the search page, showing other casualties with the same middle names.

\test{Does the casualty data link to the correct search page? (Family Name)}
Navigate back to the casualty named \texttt{Frank Aaamith}. Under the data section of this casualty, click on the family name. This should load the search page, showing other casualties with the same family name.

\test{Does the casualty data link to the correct search page? (Civilian)}
Navigate back to the casualty named \texttt{Frank Aaamith}. Under the data section of this casualty, click on the civilian status. This should load the search page, showing other casualties with the same civilian status.

\test{Does the casualty data link to the correct search page? (Gender)}
Navigate back to the casualty named \texttt{Frank Aaamith}. Under the data section of this casualty, click on the gender. This should load the search page, showing other casualties with the same gender.

\test{Does the casualty data link to the correct search page? (Date of Birth)}
Navigate back to the casualty named \texttt{Frank Aaamith}. Under the data section of this casualty, click on the Date of Birth. This should load the search page, showing other casualties with the same Date of Birth.

\test{Does the casualty data link to the correct search page? (Date of Death)}
Navigate back to the casualty named \texttt{Frank Aaamith}. Under the data section of this casualty, click on the Date of Death. This should load the search page, showing other casualties with the same Date of Death.

\test{Does the casualty data link to the correct search page? (War)}
Navigate back to the casualty named \texttt{Frank Aaamith}. Under the data section of this casualty, click on the War. This should load the search page, showing other casualties with the same War.

\test{Does the casualty data link to the correct search page? (Final Resting Place)}
Navigate back to the casualty named \texttt{Frank Aaamith}. Under the data section of this casualty, click on the Final Resting Place. This should load the search page, showing other casualties with the same Final Resting Place.

\test{Does the casualty data link to the correct search page? (Rank)}
Navigate back to the casualty named \texttt{Frank Aaamith}. Under the data section of this casualty, click on the Rank. This should load the search page, showing other casualties with the same Rank.

\test{Does the casualty data link to the correct search page? (Service Country)}
Navigate back to the casualty named \texttt{Frank Aaamith}. Under the data section of this casualty, click on the Service Country. This should load the search page, showing other casualties with the same Service Country.

\test{Does the casualty data link to the correct search page? (Place of Birth)}
Navigate back to the casualty named \texttt{Frank Aaamith}. Under the data section of this casualty, click on the Place of Birth. This should load the search page, showing other casualties with the same Place of Birth.

\test{Does the casualty data link to the correct search page? (Last Known Address)}
Navigate back to the casualty named \texttt{Frank Aaamith}. Under the data section of this casualty, click on the Last Known Address. This should load the search page, showing other casualties with the same Last Known Address.

\test{Does the casualty data link to the correct search page? (Last Known Address Year)}
Navigate back to the casualty named \texttt{Frank Aaamith}. Under the data section of this casualty, click on the Last Known Address Year. This should load the search page, showing other casualties with the same Last Known Address Year.

\test{Does the casualty data link to the correct search page? (Commemoration Location)}
Navigate back to the casualty named \texttt{Frank Aaamith}. Under the data section of this casualty, click on the Commemoration Location. This should load the search page, showing other casualties with the same Commemoration Location.

\test{Does the casualty data link to the correct search page? (Regiment / Service)}
Navigate back to the casualty named \texttt{Frank Aaamith}. Under the data section of this casualty, click on the Regiment / Service. This should load the search page, showing other casualties with the same Regiment / Service.

\test{Does the casualty data link to the correct search page? (Service Numbers)}
Navigate back to the casualty named \texttt{Frank Aaamith}. Under the data section of this casualty, click on a Service Number. This should load the search page, showing other casualties with the same Service Number.

\section{Other Tests}\label{sec:other}

There are other tests that are required. For example, the listing, editing and deletion of places, ranks, etc. Other tests involve the buttons to create narratives with various styling. The styling is created using a system called Markdown that allows much easier notation than HTML, which also not potentially breaking other sections of the page if incorrect. Having buttons to edit the style should make it easier than modifying any code.

If you are not logged in, do so now.

\test{Is the list of Places displayed correctly?}
Navigate to the list of Places by clicking on the \textit{Config \& Meta Pages} link on the top menu bar and clicking on \textit{List All} next to \texttt{Place}. A list of Places should be displayed.

\test{Can a Place be added?}
Next to the title, there should be an \textit{New} button. Click on this. Enter a name, latitude and longitude and click on \textit{Save place}.

\test{Has the Place been added correctly? (1)}
The list of places should reload with the newly created place.

\test{Has the Place been added correctly? (2)}
Navigate to the search page using the \textit{Search} button on either the top or bottom menu bar. In the data section, view the options for the \texttt{Final Resting Place} field. The newly created Place should be displayed in this list.

\test{Can a Place be edited?}
Navigate back to the list of Places using the method described above. Click the \textit{Edit} button. Change the \texttt{Name} field and click \textit{Save place} button. The list of Places should be displayed and the newly edited Place should display its new Name.

\test{Can a Place be deleted?}
Click the \textit{Edit} button to the newly created Place. The edit page should load. In the top right corner, click the \textit{Delete Place} button. Confirm you wish to delete the Place. The list of Places should be displayed again, with the newly deleted Place not displayed.

\test{Is the list of Ranks displayed correctly?}
Navigate to the list of Ranks by clicking on the \textit{Config \& Meta Pages} link on the top menu bar and clicking on \textit{List All} next to \texttt{Rank}. A list of Ranks should be displayed.

\test{Can a Rank be added?}
Next to the title, there should be an \textit{New} button. Click on this. Enter a name and click on \textit{Save Rank}.

\test{Has the Rank been added correctly? (1)}
The list of Ranks should reload with the newly created place.

\test{Has the Rank been added correctly? (2)}
Navigate to the search page using the \textit{Search} button on either the top or bottom menu bar. In the data section, view the options for the \texttt{Rank} field. The newly created Rank should be displayed in this list.

\test{Can a Rank be edited?}
Navigate back to the list of Ranks using the method described above. Click the \textit{Edit} button. Change the \texttt{Name} field and click \textit{Save Rank} button. The list of Ranks should be displayed and the newly edited Place should display its new Name.

\test{Can a Rank be deleted?}
Click the \textit{Edit} button to the newly created Rank. The edit page should load. In the top right corner, click the \textit{Delete Rank} button. Confirm you wish to delete the Rank. The list of Ranks should be displayed again, with the newly deleted Rank not displayed.

\test{Is the list of Regiment / Services displayed correctly?}
Navigate to the list of Regiment / Services by clicking on the \textit{Config \& Meta Pages} link on the top menu bar and clicking on \textit{List All} next to \texttt{Regiment / Service}. A list of Regiment / Services should be displayed.

\test{Can a Regiment / Service be added?}
Next to the title, there should be an \textit{New} button. Click on this. Enter a name and click on \textit{Save Regiment / Service}.

\test{Has the Regiment / Service been added correctly? (1)}
The list of Regiment / Services should reload with the newly created place.

\test{Has the Regiment / Service been added correctly? (2)}
Navigate to the search page using the \textit{Search} button on either the top or bottom menu bar. In the data section, view the options for the \texttt{Regiment / Service} field. The newly created Regiment / Service should be displayed in this list.

\test{Can a Regiment / Service be edited?}
Navigate back to the list of Regiment / Services using the method described above. Click the \textit{Edit} button. Change the \texttt{Name} field and click \textit{Save Regiment / Service} button. The list of Regiment / Services should be displayed and the newly edited Place should display its new Name.

\test{Can a Regiment / Service be deleted?}
Click the \textit{Edit} button to the newly created Regiment / Service. The edit page should load. In the top right corner, click the \textit{Delete Regiment / Service} button. Confirm you wish to delete the Regiment / Service. The list of Regiment / Services should be displayed again, with the newly deleted Regiment / Service not displayed.

\test{Is the list of Relation Types displayed correctly?}
Navigate to the list of Relation Types by clicking on the \textit{Config \& Meta Pages} link on the top menu bar and clicking on \textit{List All} next to \texttt{Relation Type}. A list of Relation Types should be displayed.

\test{Can a Relation Type be added?}
Next to the title, there should be an \textit{New} button. Click on this. Enter a name and click on \textit{Save Relation Type}.

\test{Has the Relation Type been added correctly? (1)}
The list of Relation Types should reload with the newly created place.

\test{Has the Relation Type been added correctly? (2)}
Navigate to the search page using the \textit{Search} button on either the top or bottom menu bar. In the data section, view the options for the \texttt{Relation Type} field. The newly created Relation Type should be displayed in this list.

\test{Can a Relation Type be edited?}
Navigate back to the list of Relation Types using the method described above. Click the \textit{Edit} button. Change the \texttt{Name} field and click \textit{Save Relation Type} button. The list of Relation Types should be displayed and the newly edited Place should display its new Name.

\test{Can a Relation Type be deleted?}
Click the \textit{Edit} button to the newly created Relation Type. The edit page should load. In the top right corner, click the \textit{Delete Relation Type} button. Confirm you wish to delete the Relation Type. The list of Relation Types should be displayed again, with the newly deleted Relation Type not displayed.

\test{Is the list of Service Countrys displayed correctly?}
Navigate to the list of Service Countrys by clicking on the \textit{Config \& Meta Pages} link on the top menu bar and clicking on \textit{List All} next to \texttt{Service Country}. A list of Service Countrys should be displayed.

\test{Can a Service Country be added?}
Next to the title, there should be an \textit{New} button. Click on this. Enter an ID and a name and click on \textit{Save Service Country}. The ID must be an ISO3166-alpha 2 country code, but if a historical country, use any codes between QM-QZ and XA-XZ. All countries should already be added.

\test{Has the Service Country been added correctly? (1)}
The list of Service Countrys should reload with the newly created place.

\test{Has the Service Country been added correctly? (2)}
Navigate to the search page using the \textit{Search} button on either the top or bottom menu bar. In the data section, view the options for the \texttt{Service Country} field. The newly created Service Country should be displayed in this list.

\test{Can a Service Country be edited?}
Navigate back to the list of Service Countrys using the method described above. Click the \textit{Edit} button. Change the \texttt{Name} field and click \textit{Save Service Country} button. The list of Service Countrys should be displayed and the newly edited Place should display its new Name.

\test{Can a Service Country be deleted?}
Click the \textit{Edit} button to the newly created Service Country. The edit page should load. In the top right corner, click the \textit{Delete Service Country} button. Confirm you wish to delete the Service Country. The list of Service Countrys should be displayed again, with the newly deleted Service Country not displayed.

\test{Is the list of Wars displayed correctly?}
Navigate to the list of Wars by clicking on the \textit{Config \& Meta Pages} link on the top menu bar and clicking on \textit{List All} next to \texttt{War}. A list of Wars should be displayed.

\test{Can a War be added?}
Next to the title, there should be an \textit{New} button. Click on this. Enter a name and click on \textit{Save War}.

\test{Has the War been added correctly? (1)}
The list of Wars should reload with the newly created place.

\test{Has the War been added correctly? (2)}
Navigate to the search page using the \textit{Search} button on either the top or bottom menu bar. In the data section, view the options for the \texttt{War} field. The newly created War should be displayed in this list.

\test{Can a War be edited?}
Navigate back to the list of Wars using the method described above. Click the \textit{Edit} button. Change the \texttt{Name} field and click \textit{Save War} button. The list of Wars should be displayed and the newly edited Place should display its new Name.

\test{Can a War be deleted?}
Click the \textit{Edit} button to the newly created War. The edit page should load. In the top right corner, click the \textit{Delete War} button. Confirm you wish to delete the War. The list of Wars should be displayed again, with the newly deleted War not displayed.

\test{Can narrative text be made bold?}
The following tests ensure that styling can be applied to narrative sections of the site. Navigate to the page to create a new site update. This will be the area tested - the narrative editing tools are shared across the site.

Enter some text into the box. Highlight this text and click on the \textit{B} button. The text should be surrounded by \texttt{\**\**}.

\test{Can narrative text be previewed?}
Click the \textit{Preview Below} button. The text entered should be displayed below and in bold type.

\test{Can narrative text be made italicised?}
Delete the text currently in the narrative area and type some other text in. Highlight this and click the \textit{I} button. The text should be surrounded by \texttt{\textunderscore}. Click the \textit{Preview Below} button. The text entered should be displayed below and in italics.

\test{Can narrative text be made into a header?}
Delete the text currently in the narrative area and type some other text in. Highlight this and click the \textit{H} button. The text should be prefixed by \texttt{\#\#\#}. Click the \textit{Preview Below} button. The text entered should be displayed below and in the header style.

\test{Can a link be added to narrative text?}
Delete the text currently in the narrative area and type some other text in. Highlight this text and click the link button. Type the link into the box that appears and click \textit{OK}. The text now becomes a link. Click the \textit{Preview Below} button. The text entered should be displayed below and as a link.

\test{Can a bullet point list be added to narrative text?}
Delete the text currently in the narrative area and type some other text in over several lines. Highlight this text and click the bullet list button. The text should be prefixed by \texttt{-}. Click the \textit{Preview Below} button. The text entered should be displayed below and as a bullet point list.

\test{Can a numbered list be added to narrative text?}
Delete the text currently in the narrative area and type some other text in over several lines. Highlight this text and click the bullet list button. The text should be prefixed by \texttt{1.}. Click the \textit{Preview Below} button. The text entered should be displayed below and as a numbered list.

\test{Can a quote be added to narrative text?}
Delete the text currently in the narrative area and type some other text in. Highlight this and click the \textit{``} button. The text should be prefixed by \texttt{\textgreater}. Click the \textit{Preview Below} button. The text entered should be displayed below and in the quotation style.

\test{Can a picture be added to narrative text? (Left)}
Delete the text currently in the narrative area. Click the \textit{Left} button. In the window that appears, type \texttt{SurnamesS/SamwaysAH.gif} and click \textit{OK}. In the next window, provide a description of that picture and click \textit{OK}. The code for a picture should be added. Click the \textit{Preview Below} button. The preview area should show the picture aligned to the left. Hover over the picture to show the description provided.

\test{Can a picture be added to narrative text? (Center)}
Delete the text currently in the narrative area. Click the \textit{Center} button. In the window that appears, type \texttt{SurnamesS/SamwaysAH.gif} and click \textit{OK}. In the next window, provide a description of that picture and click \textit{OK}. The code for a picture should be added. Click the \textit{Preview Below} button. The preview area should show the picture aligned in the middle. Hover over the picture to show the description provided.

\test{Can a picture be added to narrative text? (Right)}
Delete the text currently in the narrative area. Click the \textit{Right} button. In the window that appears, type \texttt{SurnamesS/SamwaysAH.gif} and click \textit{OK}. In the next window, provide a description of that picture and click \textit{OK}. The code for a picture should be added. Click the \textit{Preview Below} button. The preview area should show the picture aligned to the right. Hover over the picture to show the description provided.

\test{Can the narrative area be maximised?}
In the top right corner, of the narrative area, there should be a button to maximise the narrative window. Click this, and the page should be replaces with a large text area. Clicking the button again should return the narrative window to its normal size.

\test{Does the narrative help page load correctly?}
Slightly below the narrative window, there should be a link that indicates help is available for Markdown. Clicking on this should load a new tab with a help page on how to use the narrative area.

\end{document}
